Collecting a high-quality dataset is a critical task that demands meticulous attention to detail, as overlooking certain aspects can render the entire dataset unusable. Autonomous driving challenges remain a prominent area of research, requiring further exploration to enhance the perception and planning performance of vehicles. However, existing datasets are often incomplete. For instance, datasets that include perception information generally lack planning data, while planning datasets typically consist of extensive driving sequences where the ego vehicle predominantly drives forward, offering limited behavioral diversity.
The CARLA Leaderboard 2.0 challenge, providing a diverse set of scenarios to address the long-tail problem in autonomous driving, has emerged as a valuable alternative platform for developing perception and planning models. Nevertheless, existing datasets collected on this platform present certain limitations. Some datasets appear to be tailored primarily for limited sensor configuration, with particular sensor configurations. Additionally, in some datasets, the expert policies used for data collection exhibit suboptimal driving behaviors, such as oscillations. 
To support end-to-end autonomous driving research, we have collected a new dataset comprising over 2.85 million frames using the CARLA simulation environment for the diverse Leaderboard 2.0 challenge scenarios, making it the largest dataset in the literature to the best of our knowledge. Our dataset is designed not only for planning tasks but also supports dynamic object detection, lane divider detection, centerline detection, traffic light recognition, and prediction tasks. Furthermore, we demonstrate its versatility by training various models using our dataset. 